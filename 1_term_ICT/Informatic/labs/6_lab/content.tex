\section{Текст лабораторной}
\begin{enumerate} 
\item Перевести число  ``А'' , заданное в системе счисления ``В'', в систему счисления ``С''. Числа ``A'', ``В'' и ``С'' взять из представленных ниже таблиц.
\item Всего нужно решить 13 примеров. Для примеров с 5-го по 7-й выполнить операцию перевода по сокращенному правилу (для систем с основанием 2 в системы с основанием $2^\wedge k$). В примере 11 группа символов ${\widehat{1}}$ означает -1 в симметричной системе счисления.
\end{enumerate}
\section{Выполнение заданий}
\paragraph{Задание 1}


$85407_{10} \rightarrow X_{11}$


\paragraph{Основные этапы вычисления:}
\hfill \break

\FloatBarrier
\begin{table}[h]
\centering
\begin{tabular}{lllll}
\multicolumn{1}{l|}{85407} & 11                        &                          &                         &     \\ 
\cline{2-2}
\multicolumn{1}{l|}{85404} & \multicolumn{1}{l|}{7764} & 11                       &                         &     \\ 
\cline{1-1}\cline{3-3}
3                          & \multicolumn{1}{l|}{7755} & \multicolumn{1}{l|}{705} & 11                      &     \\ 
\cline{2-2}\cline{4-4}
                           & 9                         & \multicolumn{1}{l|}{704} & \multicolumn{1}{l|}{64} & 11  \\ 
\cline{3-3}\cline{5-5}
                           &                           & 1                        & \multicolumn{1}{l|}{55} & 5   \\ 
\cline{4-4}
                           &                           &                          & 9                       &     \\
                           &                           &                          &                         &    
\end{tabular}
\end{table}
\FloatBarrier
$85407_{10} \rightarrow 59193_{11}$
\paragraph{Ответ:}
$59193_{11}$
\paragraph{Задание 2}
$1A550_{11} \rightarrow X_{10}$
\paragraph{Основные этапы вычисления:}
\hfill \break
\hfill \break
$1A550_{11}=1\cdot 11^4 + 10\cdot 11^3 +5\cdot 11^2 + 5\cdot11+0=14641+13310+605+55=28611_{10}$

\paragraph{Ответ:}
$28611_{10}$

\paragraph{Задание 3}
$43455_{7} \rightarrow X_{13}$

\paragraph{Основные этапы вычисления:}

\begin{enumerate}
\item Переведём $43455_{7} \rightarrow X_{10}$

$43455_{7} = 4\cdot7^4 + 3\cdot 7^3 + 4\cdot 7^2 + 5\cdot7+5=9604+1029+196+35+5=10869_{10}$

\item Переведём $10869_{10} \rightarrow X_{13}$

\FloatBarrier
\begin{table}
\centering
\begin{tabular}{lllll}
\multicolumn{1}{l|}{10869} & 13                       &                         &    &   \\ 
\cline{2-2}
\multicolumn{1}{l|}{10868} & \multicolumn{1}{l|}{836} & 13                      &    &   \\ 
\cline{1-1}\cline{3-3}
1                          & \multicolumn{1}{l|}{832} & \multicolumn{1}{l|}{64} & 13 &   \\ 
\cline{2-2}\cline{4-4}
                           & 4                        & \multicolumn{1}{l|}{52} & 4  &   \\ 
\cline{3-3}
                           &                          & С                       &    &   \\
                           &                          &                         &    &   \\
                           &                          &                         &    &  
\end{tabular}
\end{table}
\FloatBarrier
$43455_{7} \rightarrow 4C41_{13}$
\end{enumerate}

\paragraph{Ответ:}
$4C41_{13}$

\paragraph{Задание 4}
$36,19_{10} \rightarrow X_{2}$
\paragraph{Основные этапы вычисления:}
\begin{enumerate}
\item Переведём $36_{10} \rightarrow X_{2}$

\FloatBarrier
\begin{table}[bh]
\centering
\begin{tabular}{llllll}
\multicolumn{1}{l|}{36} & 2                       &                        &                        &                        &    \\ 
\cline{2-2}
\multicolumn{1}{l|}{36} & \multicolumn{1}{l|}{18} & 2                      &                        &                        &    \\ 
\cline{1-1}\cline{3-3}
0                       & \multicolumn{1}{l|}{18} & \multicolumn{1}{l|}{9} & 2                      &                        &    \\ 
\cline{2-2}\cline{4-4}
                        & 0                       & \multicolumn{1}{l|}{8} & \multicolumn{1}{l|}{4} & 2                      &    \\ 
\cline{3-3}\cline{5-5}
                        &                         & 1                      & \multicolumn{1}{l|}{4} & \multicolumn{1}{l|}{2} & 2  \\ 
\cline{4-4}\cline{6-6}
                        &                         &                        & 0                      & \multicolumn{1}{l|}{2} & 1  \\ 
\cline{5-5}
                        &                         &                        &                        & 0                      &   
\end{tabular}
\end{table}
\FloatBarrier
$36_{10} \rightarrow 100100_{2}$
\item Переведём $0,19_{10} \rightarrow X_{2}$
\FloatBarrier
\begin{table}[bh]
\centering
\begin{tabular}{|l|l|llll}
0 & ,19 &  &  &  &   \\
  & 2   &  &  &  &   \\ 
\cline{1-2}
0 & ,38 &  &  &  &   \\
  & 2   &  &  &  &   \\ 
\cline{1-2}
0 & ,76 &  &  &  &   \\
  & 2   &  &  &  &   \\ 
\cline{1-2}
1 & ,52 &  &  &  &   \\
  & 2   &  &  &  &   \\ 
\cline{1-2}
1 & ,04 &  &  &  &   \\
  & 2   &  &  &  &   \\ 
\cline{1-2}
0 & ,08 &  &  &  &   \\
  & 2   &  &  &  &  
\end{tabular}
\end{table}
\FloatBarrier
$0,19_{10} \rightarrow 0,0011_{2}$
\item Тогда $36,19_{10} \rightarrow 100100,0011_{2}$
\end{enumerate}
\paragraph{Ответ:}
$100100,0011_{2}$

\paragraph{Задание 5}
$83,E1_{16} \rightarrow X_{2}$
\paragraph{Основные этапы вычисления:}
\hfill \break


Зная, что:

$8_{16} <=> 1000_{2}$;

$3_{16} <=> 0011_{2}$;

$E_{16} <=> 1110_{2}$;

$1_{16} <=> 0001_{2}$;

Выполним перевод:

$83,E1_{16} = 1000~0011,~1110~0001_{2}$

\paragraph{Ответ:}
$10000011,111_{2}$

\paragraph{Задание 6}
$22,32_{8} \rightarrow X_{2}$
\paragraph{Основные этапы вычисления:}
\hfill \break


Зная, что:

$2_{8} <=> 010_{2}$;

$3_{8} <=> 011_{2}$;

Выполним перевод:

$22,32_{8} = 10~010,~011~01_{2}$

\paragraph{Ответ:}
$10010,01101_{2}$

\paragraph{Задание 7}
$0,011101_{2} \rightarrow X_{16}$
\paragraph{Основные этапы вычисления:}
\hfill \break


Зная, что:

$7_{16} <=> 0111_{2}$;

$4_{8} <=> 0100_{2}$;

Выполним перевод:

$0,0111~0100_{2} = 0,74_{16}$

\paragraph{Ответ:}
$0,74_{16}$

\paragraph{Задание 8}
$0,011101_{2} \rightarrow X_{10}$

\paragraph{Основные этапы вычисления:}
\hfill \break
$0,011101_{2} = 0 + 0\cdot2{-1} + 1\cdot2^{-2} + 1\cdot2^{-3} + 1\cdot2^{-4} + 0\cdot2^{-5}+1\cdot2^{-6} = 0,25+0.125+0,0625+0,015625 = 0,453125 \approx 0,45313_{10}$

\paragraph{Ответ:}
$0,45313_{10}$

\paragraph{Задание 9}
$B7,F4_{16} \rightarrow X_{10}$

\paragraph{Основные этапы вычисления:}
\hfill \break
$B7,F4_{16} = 11\cdot16+7+15\cdot16^{-1}+4\cdot16^{-2} = 176+7+\frac{15}{16}+\frac{4}{256} =\frac{11773}{64} \approx 183,95313_{10}$
\paragraph{Ответ:}
$183,95313_{10}$

\paragraph{Задание 10}
$67_{10} \rightarrow X_\text{Фиб}$

\paragraph{Основные этапы вычисления:}
\hfill \break

$67_{10} = 55 + 8 + 3 + 1 = 100010101_\text{Фиб}$

\paragraph{Ответ:}
$100010101_\text{Фиб}$

\paragraph{Задание 11}
$692_{-10} \rightarrow X_{10}$

\paragraph{Основные этапы вычисления:}
\hfill \break

$692_{-10} = 6\cdot(-10)^{2}+9\cdot(-10)+2 = 600 - 90 + 2 = 512_{10}$

\paragraph{Ответ:}
$512_{10}$

\paragraph{Задание 12}
$32\overline{3}44_\text{9C} \rightarrow X_{10}$

\paragraph{Основные этапы вычисления:}
\hfill \break

$32\overline{3}44_\text{9C} = 3\cdot9^{4}+2\cdot9^{3}+(-3)\cdot9^{2}+4\cdot9+4=20938_{10}$

\paragraph{Ответ:}
$20938_{10}$

\paragraph{Задание 13}
$3088_{10} \rightarrow X_\text{Ф}$

\paragraph{Основные этапы вычисления:}
\hfill \break
\begin{enumerate}
\item 3088 div 2 = 1544
\item 1544 div 3 = 514
\item 514 div 4 = 128
\item 128 div 5 = 25
\item 25 div 6 = 4
\item 4 div 7 = 0
\end{enumerate}
$X_\text{Ф} = d_1d_2d_3d_4d_5d_\text{6(Ф)}$
\begin{enumerate}
\item $d_{1}$ = 3088 mod 2 = 0
\item $d_{2}$ = 1544 mod 3 = 2
\item $d_{3}$ = 514 mod 4 = 2
\item $d_{4}$ = 128 mod 5 = 3
\item $d_{5}$ = 25 mod 6 = 1
\item $d_{6}$ = 4 mod 7 = 4
\end{enumerate}
$3088_{10} \rightarrow 413220_\text{Ф}$

\paragraph{Ответ:}
$413220_\text{Ф}$
\section{Вывод}
В этой лабораторной работе я научился:
\begin{itemize}
\item Переводить из СС-10 в любую другую СС;
\item Переводить из любой СС в СС-10;
\item Переводить по сокращённому правилу для(для систем с основанием 2 в системы с основанием 2 $\widehat{}$ k);
\item Переводить из СС-10 в факториальную СС;
\item Переводить из нега-позиционных СС, Симметричных СС, CC Цекендорфа в СС-10;
\end{itemize}